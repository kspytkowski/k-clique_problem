\documentclass[11pt]{aghdpl}
\usepackage[polish]{babel}
\usepackage[utf8]{inputenc}
\usepackage{enumerate}

\author{Wojciech Kasperek, Krzysztof Spytkowski, Izabela Śmietana}
\shortauthor{W. Kasperek, K. Spytkowski, I. Śmietana}

\titlePL{Problem istnienia k-kliki}
\titleEN{}

\thesistype{Projekt zaliczeniowy:}
%\thesistype{Master of Science Thesis}

\supervisor{dr Adam Sędziwy}
%\supervisor{Marcin Szpyrka PhD, DSc}

\degreeprogramme{Informatyka}
%\degreeprogramme{Computer Science}

\date{2014}

\department{Katedra Informatyki Stosowanej}
%\department{Department of Applied Computer Science}

\faculty{Wydział Elektrotechniki, Automatyki,\protect\\[-1mm] Informatyki i Inżynierii Biomedycznej}
%\faculty{Faculty of Electrical Engineering, Automatics, Computer Science and Biomedical Engineering}
%\acknowledgements{Serdecznie dziękuję \dots tu ciąg dalszych podziękowań np. dla promotora, żony, sąsiada itp.}
\setlength{\cftsecnumwidth}{10mm}

%---------------------------------------------------------------------------

\begin{document}

\titlepages

\tableofcontents
\clearpage

\chapter{Zarys problemu}
\label{cha:wprowadzenie}
PARAPAPA o problemie (o grafach złożoności NP cośtam etc.)
\section{Algorytmy genetyczne}
\label{sec:algGenetyczne}
o algorytmach ogólnie słów parę
\section{Podejście do rozwiązania}
\label{sec:podejscie}
bla bla bla, że mieliśmy dwa pomysły z kodowaniami
że teraz wyjaśnimi co i jak, a później w jakimśtam rozdziale będzie jak przebiega całość

\chapter{Kodowania}
\label{cha:encoding}
\section{Binarne}
\label{sec:binary}
to takie wolne o.O
\section{Grupowe}
\label{sec:group}
takie pro

\chapter{Metody krzyżowania}
\label{cha:crossing}
\section{Jednopunktowe}
\label{sec:singlePoint}
ho
\section{Uniform??? + ważony}
\label{sec:uniform}
fd
\section{Dwupunktowe}
\label{sec:twoPoints}
d

\chapter{Metody selekcji}
\label{cha:selection}
\section{Turniejowa}
\label{sec:tournament}
taka fajna
\section{Ruletki}
\label{sec:roulette}
taka niebezpieczna
\section{Rankingowa}
\label{sec:linear}
a co to to nie wiem

\chapter{Mutowanie}
\label{cha:mutation}

\chapter{Działanie programu}
\label{cha:program}
\section{Przebieg pojedynczej iteracji}
\label{sec:singleLifeCycle}
singleLifeCycle
\section{Parametry}
\label{sec:params}
dddjjdjf
co można ustalać i co to da
\section{Wykres}
\label{sec:chart}
ffd
co jest na wykresach i dlaczego
\section{Wizualizacja}
\label{sec:visualization}
jak wygląda rozwiązanie (o niebieskich krawędziach i żółtych wierzchołkach)

\chapter{O autorach} 
\label{cha:aboutAut}
podział pracy


%% tu nie moje.
%\include{rozdzial1}
%\include{rozdzial2}



% itd.
% \appendix
% \include{dodatekA}
% \include{dodatekB}
% itd.

%\bibliographystyle{alpha}
%\bibliography{bibliografia}
%\begin{thebibliography}{1}
%
%\bibitem{Dil00}
%A.~Diller.
%\newblock {\em LaTeX wiersz po wierszu}.
%\newblock Wydawnictwo Helion, Gliwice, 2000.
%
%\bibitem{Lam92}
%L.~Lamport.
%\newblock {\em LaTeX system przygotowywania dokumentów}.
%\newblock Wydawnictwo Ariel, Krakow, 1992.
%
%\bibitem{Alvis2011}
%M.~Szpyrka.
%\newblock {\em {On Line Alvis Manual}}.
%\newblock AGH University of Science and Technology, 2011.cccccc
%\newblock \\\texttt{http://fm.ia.agh.edu.pl/alvis:manual}.
%
%\end{thebibliography}

\end{document}