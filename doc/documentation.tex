\documentclass[11pt]{aghdpl}
\usepackage[polish]{babel}
\usepackage[utf8]{inputenc}
\usepackage{enumerate}
\usepackage{listings}

\author{Wojciech Kasperek, Krzysztof Spytkowski, Izabela Śmietana}
\shortauthor{W. Kasperek, K. Spytkowski, I. Śmietana}

\titlePL{Problem istnienia k-kliki}
\titleEN{}

\shorttitlePL{Problem istnienia k-kliki}
\shorttitleEN{}

\thesistype{Projekt zaliczeniowy:}
%\thesistype{Master of Science Thesis}

\supervisor{dr Adam Sędziwy}
%\supervisor{Marcin Szpyrka PhD, DSc}

\degreeprogramme{Informatyka}
%\degreeprogramme{Computer Science}

\subject{Badania operacyjne i teoria złożoności obliczeniowej}

\date{2014}

\department{Katedra Informatyki Stosowanej}
%\department{Department of Applied Computer Science}

\faculty{Wydział Elektrotechniki, Automatyki,\protect\\[-1mm] Informatyki i Inżynierii Biomedycznej}
%\faculty{Faculty of Electrical Engineering, Automatics, Computer Science and Biomedical Engineering}
%\acknowledgements{Serdecznie dziękuję \dots tu ciąg dalszych podziękowań np. dla promotora, żony, sąsiada itp.}
%%\setlength{\cftsecnumwidth}{10mm}

%---------------------------------------------------------------------------

\begin{document}

\titlepages

\tableofcontents
\clearpage

\chapter{Zarys problemu}
\label{cha:wprowadzenie}
Nasz projekt polegał na stworzeniu programu umożliwiającego stwierdzenie czy istnieje klika o zadanym rozmiarze w podanym grafie. Postawiony problem jest w teorii złożoności obliczeniowej jednym z problemów NP-zupełnych. Dlatego też nasz program do rozwiązania tego problemu korzysta z algorytmu genetycznego. 
\section{Algorytm genetyczny}
\label{sec:algGenetyczne}
Mechanizm działania klasycznego algorytmu genetycznego składa się z następujących kroków:
\begin{enumerate}
\item Inicjacja – utworzenie populacji początkowej, wybór ustalonej liczby osobników i nadanie wartości losowych genom wchodzącym w skład ich chromosomu.
\item Ocena przystosowania – obliczenie wartości funkcji przystosowania dla każdego osobnika.
\item Selekcja chromosomów – wybór osobników, które biorą udział w tworzeniu nowej populacji.
\item Zastosowanie operatorów genetycznych – na grupie osobników, wybranych drogą selekcji działają
operatory genetyczne (krzyżowanie i mutacja).
\item Utworzenie nowej populacji – osobniki otrzymane jako rezultat działania operatorów
genetycznych wchodzą w skład nowej populacji. Cała poprzednia populacja jest zastępowana przez
tak samo liczną nowa populacje potomków.
\item Sprawdzenie warunku stopu algorytmu - wyprowadzenie 'najlepszego' osobnika (o największej wartości funkcji przystosowania). Jeżeli osobnik ten spełnia postawione w zadaniu warunki to algorytm kończy pracę, w przeciwnym razie przechodzimy w pętli do punktu 2 (działając już na nowej populacji - potomków).

\end{enumerate}
\section{Podejście do rozwiązania}
\label{sec:podejscie}
bla bla bla, że mieliśmy dwa pomysły z kodowaniami
że teraz wyjaśnimi co i jak, a później w jakimśtam rozdziale będzie jak przebiega całość

\chapter{Kodowanie i funkcja przystosowania}
\label{cha:encoding}
Każdy osobnik (podgraf zadanego grafu) reprezentowany jest przez chromosom, który składa się z genów (zbioru wierzchołków podgrafu). Powstały zbiór cech danego osobnika stanowi jego genotyp, który podlega ocenie funkcji przystosowania. A więc genotyp opisuje proponowane rozwiązanie problemu, a funkcja przystosowania ocenia, jak
dobre jest to rozwiązanie.
\section{Binarne}
\label{sec:binary}
W tym kodowaniu zasada jest następująca: '0' w chromosomie oznacza przynależność danego wierzchołka do podgrafu, '1' oznacza, że tego wierzchołka nie ma w podgrafie.
\section{Grupowe}
\label{sec:group}
takie pro

\chapter{Krzyżowanie}
\label{cha:crossing}
Zadaniem krzyżowania jest wymiana 'materiału genetycznego' pomiędzy dwoma osobnikami w populacji. Polega na połączeniu niektórych (wybranych od rodziców) genotypów w jeden. Kojarzenie ma sprawić, że potomek dwóch osobników rodzicielskich ma zespół cech, który jest kombinacją ich cech (może się zdarzyć, że tych najlepszych). Po wykonaniu krzyżowania dzieci zastępują w populacji rodziców. 
 
\section{Jednopunktowe z dwoma potomkami}
\label{sec:jedenDwa}
Polega na podziale dwóch chromosomów (pochodzących od rodziców) na dwie części (niekoniecznie równe) i z nich tworzone są dzieci: pierwsze dziecko składa się z początkowej części pierwszego rodzica i końcówki drugiego natomiast drugie dziecko odwrotnie - początek drugiego rodzica i koniec pierwszego.

\section{Jednopunktowe z jednym potomkiem}
\label{sec:jedenJeden}
Krzyżowanie analogiczne do powyższego, ale w wyniku powstaje jedno dziecko, końcówka materiału genetycznego pierwszego rodzica i początek materiału genetycznego drugiego rodzica są zaniedbywane.

\section{Jednorodne z jednym potomkiem}
\label{sec:uniform}
W tym krzyżowaniu każdy gen dziecka jest losowany od rodziców z prawdopodobieństwem 50\%.

\section{Ważone z jednym potomkiem}
\label{sec:wazone}
W wyniku tego krzyżowania powstaje jeden nowy osobnik, który dziedziczy każdy gen zgodnie z prawdopodobieństwem:
\begin{center}
przystosowanie konkretnego rodzica / suma przystosowań obu rodziców
\end{center}

\section{Dwupunktowe z dwoma potomkami}
\label{sec:dwaDwa}
Tym razem dzielimy chromosomy rodziców na 3 części. Pierwsze dziecko składa się z początkowej części pierwszego rodzica, środkowej drugiego rodzica i końcówki ponownie pierwszego rodzica. Natomiast drugie dziecko odwrotnie - początek drugiego rodzica, środek pierwszego i koniec znów drugiego.

\section{Dwupunktowe z jednym potomkiem}
\label{sec:dwaJeden}
Krzyżowanie analogiczne do powyższego, ale w wyniku powstaje jedno dziecko, środek materiału genetycznego pierwszego rodzica, początek i koniec materiału genetycznego drugiego rodzica są zaniedbywane.

\chapter{Selekcja}
\label{cha:selection}
W zależności od wartości funkcji przystosowania danego osobnika w populacji ma on większe (gdy jest 'dobry') lub mniejsze (gdy jest 'słaby') szanse na znalezienie się w kolejnym pokoleniu. To właśnie selekcja dokonuje wyboru osobników będących rodzicami dla nowego pokolenia.
\section{Turniejowa}
\label{sec:tournament}
Polega na podziale populacji na grupy n-osobników (w naszej implementacji są to grupy 3-osobnikowe). Następnie z każdej z tych grup wybierany jest osobnik najlepiej przystosowany i on przepisywany jest do nowo tworzonej populacji.

\section{Ruletkowa}
\label{sec:roulette}
Polega na n krotnym losowaniu (n - liczba osobników w populacji) ze starej populacji osobników, które zostaną przepisane do nowej populacji. Oczywiście wszystkie osobniki mają różne prawdopodobieństwa wylosowania, które są liczone zgodnie ze wzorem:
\begin{center}
wartość przystosowania danego osobnika / suma wartości przystosowania wszystkich osobników
\end{center}
 
\section{Rankingowa}
\label{sec:linear}
Selekcja tą metodą jest bardzo podobna do selekcji metodą koła ruletki. Modyfikacja polega jedynie na zmianie funkcji określającej prawdopodobieństwo wyboru danego osobnika. Przed przystąpieniem do tej selekcji należy nadać każdemu z osobników pewną wartość (pozycje w rankingu) zależną od jego położenia na liście posortowanej względem wartości funkcji przystosowania. Wzór pozwalający obliczyć prawdopodobieństwo wyboru:
\begin{center}
pozycja w rankingu / suma wartości przystosowania wszystkich osobników
\end{center}

\chapter{Mutacja}
\label{cha:mutation}
Zadaniem mutacji jest wprowadzenie subtelnych zmian do genotypu losowo wybranych osobników. Dzięki temu w kolejnych pokoleniach zachowujemy różnorodność w populacji, czyli zapobiegamy przedwczesnej zbieżności algorytmu. Mutacja zachodzi z pewnym przyjętym prawdopodobieństwem - zazwyczaj rzędu 1-5\%. Jest ono niskie, ponieważ zbyt silna mutacja przynosi efekt odwrotny do zamierzonego: zamiast różnicować dobre rozwiązania - niszczy je.

\section{Przy kodowaniu binarnym}
\label{sec:kodBin}
Mutacja polega na zamianie genu „1” na gen „0” (i odwrotnie).
\section{Przy kodowaniu grupowym}
\label{sec:kodGru}

\chapter{Program}
\label{cha:program}
Stworzony przez nas program " " został napisany w języku JAVA. Do rysowania i wizualizacji grafów użyliśmy biblioteki JUNG. Przy rysowaniu i tworzeniu wykresów przystosowania osobników w populacji pomocna okazała się biblioteka JFreeChart. Natomiast zapis wykresów do pliku umożliwiła biblioteka Batik (no chyba nie zaimplementowalismy....)!
\section{Przebieg pojedynczej iteracji}
\label{sec:singleLifeCycle}
singleLifeCycle
\section{Parametry}
\label{sec:params}
dddjjdjf
co można ustalać i co to da
\section{Wykres}
\label{sec:chart}
ffd
co jest na wykresach i dlaczego
\section{Wizualizacja}
\label{sec:visualization}
jak wygląda rozwiązanie (o niebieskich krawędziach i żółtych wierzchołkach)

\chapter{O autorach} 
\label{cha:aboutAut}
podział pracy


%% tu nie moje.
%\include{rozdzial1}
%\include{rozdzial2}



% itd.
% \appendix
% \include{dodatekA}
% \include{dodatekB}
% itd.

%\bibliographystyle{alpha}
%\bibliography{bibliografia}
%\begin{thebibliography}{1}
%
%\bibitem{Dil00}
%A.~Diller.
%\newblock {\em LaTeX wiersz po wierszu}.
%\newblock Wydawnictwo Helion, Gliwice, 2000.
%
%\bibitem{Lam92}
%L.~Lamport.
%\newblock {\em LaTeX system przygotowywania dokumentów}.
%\newblock Wydawnictwo Ariel, Krakow, 1992.
%
%\bibitem{Alvis2011}
%M.~Szpyrka.
%\newblock {\em {On Line Alvis Manual}}.
%\newblock AGH University of Science and Technology, 2011.cccccc
%\newblock \\\texttt{http://fm.ia.agh.edu.pl/alvis:manual}.
%
%\end{thebibliography}

\end{document}