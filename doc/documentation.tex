\documentclass[11pt]{aghdpl}
\usepackage[polish]{babel}
\usepackage[utf8]{inputenc}
\usepackage{enumerate}
\usepackage{listings}

\author{Wojciech Kasperek, Krzysztof Spytkowski, Izabela Śmietana}
\shortauthor{W. Kasperek, K. Spytkowski, I. Śmietana}

\titlePL{Problem istnienia k-kliki}
\titleEN{}

\shorttitlePL{Problem istnienia k-kliki}
\shorttitleEN{}

\thesistype{Projekt zaliczeniowy:}
%\thesistype{Master of Science Thesis}

\supervisor{dr Adam Sędziwy}
%\supervisor{Marcin Szpyrka PhD, DSc}

\degreeprogramme{Informatyka}
%\degreeprogramme{Computer Science}

\subject{Badania operacyjne i teoria złożoności obliczeniowej}

\date{2014}

\department{Katedra Informatyki Stosowanej}
%\department{Department of Applied Computer Science}

\faculty{Wydział Elektrotechniki, Automatyki,\protect\\[-1mm] Informatyki i Inżynierii Biomedycznej}
%\faculty{Faculty of Electrical Engineering, Automatics, Computer Science and Biomedical Engineering}
%\acknowledgements{Serdecznie dziękuję \dots tu ciąg dalszych podziękowań np. dla promotora, żony, sąsiada itp.}
%%\setlength{\cftsecnumwidth}{10mm}

%---------------------------------------------------------------------------

\begin{document}

\titlepages

\tableofcontents
\clearpage

\chapter{Zarys problemu}
\label{cha:zarys}

%\section{Omówienie tematu}
%\label{sec:omowienie}

Wybrany przez nas temat dotyczy kliki, dlatego omówienie zaczniemy od~wprowadzenia tego pojęcia. 
Kliką w~grafie nazywamy zbiór wierzchołków, w~którym każda para wierzchołków jest połączona krawędzią, 
czyli podgraf będący grafem pełnym. Problem istnienia k-kliki polega na~stwierdzeniu 
czy~w~danym grafie istnieje klika o~podanym rozmiarze $k$. Problem ten należy do~klasy NP, co~oznacza, że~rozwiązanie 
można zweryfikować w czasie wielomianowym (konkretnie: mając podane wierzchołki należące do~szukanej
kliki możemy w czasie $k^2$ sprawdzić, czy faktycznie jest to klika).

%trochę za bardzo zerżnięte, nie?
Problem istnienia k-kliki jest także jednym z~pierwszych zidentyfikowanych problemów NP-zupełnych. 
Problem NP-zupełny to~problem, który należy do klasy NP oraz dowolny problem należący do~NP może być do~niego 
zredukowany w~czasie wielomianowym. NP-zupełność naszego problemu wynika z~NP-zupełności problemu 
zbioru niezależnego. Problem ten to~pytanie czy~dla danego grafu $G$ i~liczby $k$, istnieje w~$G$ zbiór 
niezależny (zbiór wierzchołków niepołączonych żadnymi krawędziami) o~$k$~wierzchołkach. W~naszym problemie 
w~grafie istnieje klika o~rozmiarze $k$~wtedy i~tylko wtedy gdy~w~dopełnieniu grafu istnieje zbiór niezależny 
o~rozmiarze $k$.

Ze~względu na~klasę złożoności problemu i~brak odpowiednio szybkich algorytmów dokładnych, w~naszym projekcie 
posłużymy się~algorytmem przybliżonym. 

\section{Klasyczny algorytm genetyczny}
\label{sec:algGenetyczne}
Postawiony problem definiuje nam środowisko, w którym istnieje pewna populacja osobników. Każdy z osobników ma 
przypisany pewien zbiór informacji stanowiących jego genotyp, a będących podstawą do utworzenia
fenotypu. Fenotyp to zbiór cech podlegających ocenie funkcji przystosowania modelującej środowisko.
Innymi słowy - genotyp opisuje proponowane rozwiązanie problemu, a funkcja przystosowania ocenia, jak
dobre jest to rozwiązanie.

Genotyp składa się z chromosomów, gdzie zakodowany jest fenotyp i ewentualnie pewne informacje
pomocnicze dla algorytmu genetycznego. Chromosom składa się z genów.

Mechanizm działania klasycznego algorytmu genetycznego składa się z następujących kroków:
\begin{enumerate}
\item Inicjacja – utworzenie populacji początkowej, wybór ustalonej liczby osobników i nadanie wartości losowych 
genom wchodzącym w skład ich chromosomów.
\item Ocena przystosowania – obliczenie wartości funkcji przystosowania dla każdego osobnika.
\item Selekcja – wybór osobników, które będą brać udział w tworzeniu nowej populacji.
\item Zastosowanie operatorów genetycznych – na grupie osobników wybranych drogą selekcji działają
operatory genetyczne (krzyżowanie i mutacja).
\item Utworzenie nowej populacji – osobniki otrzymane jako rezultat działania operatorów
genetycznych wchodzą w skład nowej populacji. Cała poprzednia populacja jest zastępowana przez
tak samo liczną nową populację potomków.
\item Sprawdzenie warunku stopu algorytmu - wyprowadzenie 'najlepszego' osobnika (o największej wartości funkcji 
przystosowania). Jeżeli osobnik ten spełnia postawione w zadaniu warunki to algorytm kończy pracę, w przeciwnym 
razie przechodzi do punktu 2 (działając już na nowej populacji - potomków).

\end{enumerate}
\section{Własna implementacja//interpretacja algorytmu}
\label{sec:podejscie}
bla bla bla, że mieliśmy dwa pomysły z kodowaniami
że teraz wyjaśnimi co i jak, a później w jakimśtam rozdziale będzie jak przebiega całość => wlasnie tutaj bedzie 
jak to ogolnie przebiega, szczegoly w nastepnym rozdziale

\chapter{Kodowanie i funkcja przystosowania}
\label{cha:encoding}
\section{Binarne}
\label{sec:binary}
W binarnym kodowaniu chromosomu zasada jest następująca: gen '0' w chromosomie oznacza przynależność danego wierzchołka 
do podgrafu, '1' oznacza, że tego wierzchołka nie ma w podgrafie.
\section{Grupowe}
\label{sec:group}
takie pro

\chapter{Selekcja}
\label{cha:selection}
Selekcja dokonuje wyboru osobników będących rodzicami dla nowego pokolenia.
W zależności od wartości funkcji przystosowania danego osobnika w populacji ma on większe (gdy jest 'dobry') lub mniejsze 
(gdy jest 'słaby') szanse na znalezienie się w kolejnym pokoleniu. 
\section{Turniejowa}
\label{sec:tournament}
Polega na podziale populacji na grupy n-osobników (w naszej implementacji są to grupy 3-osobnikowe). Następnie z każdej z 
tych grup wybierany jest najlepiej przystosowany osobnik, który przepisywany jest do nowo tworzonej populacji.

\section{Koła ruletki}
\label{sec:roulette}
Polega na n krotnym (n - liczba osobników w populacji) losowaniu osobników ze starej populacji i przepisaniu ich do nowej 
populacji. Przed rozpoczęciem losowania każdemu osobnikowi nadawane jest prawdopodobieństwo wylosowania, które wyliczane 
jest zgodnie ze wzorem:
\begin{center}
wartość przystosowania danego osobnika / suma wartości przystosowania wszystkich osobników
\end{center}
 
\section{Rankingu liniowego}
\label{sec:linear}
Selekcja tą metodą jest bardzo podobna do selekcji metodą koła ruletki. Modyfikacja polega jedynie na zmianie funkcji 
określającej prawdopodobieństwo wyboru danego osobnika. Przed przystąpieniem do tej selekcji należy nadać każdemu z 
osobników pewną wartość (pozycje w rankingu) zależną od jego położenia na liście posortowanej względem wartości funkcji 
przystosowania. Wzór pozwalający obliczyć prawdopodobieństwo wyboru to:
\begin{center}
pozycja w rankingu / suma wartości przystosowania wszystkich osobników
\end{center}

\chapter{Krzyżowanie}
\label{cha:crossing}
Zadaniem krzyżowania jest wymiana 'materiału genetycznego' pomiędzy dwoma osobnikami w populacji. Polega na połączeniu 
niektórych (wybranych od rodziców) genotypów w jeden. Kojarzenie ma sprawić, że potomek dwóch osobników rodzicielskich 
ma zespół cech, który jest kombinacją ich cech (może się zdarzyć, że tych najlepszych). 
 
\section{Jednopunktowe z dwoma potomkami}
\label{sec:jedenDwa}
Polega na podziale dwóch chromosomów (pochodzących od rodziców) na dwie części (niekoniecznie równe) i utworzeniu z nich 
dwójki dzieci: pierwsze dziecko składa się z początkowej części materiału genetycznego pierwszego rodzica i końcówki drugiego, 
natomiast drugie dziecko otrzymuje początek materiału genetycznego drugiego rodzica i końcówkę pierwszego.

\section{Jednopunktowe z jednym potomkiem}
\label{sec:jedenJeden}
Krzyżowanie analogiczne do powyższego, ale w wyniku powstaje jedno dziecko, końcówka materiału genetycznego pierwszego 
rodzica i początek materiału genetycznego drugiego rodzica są zaniedbywane.

\section{Jednorodne z jednym potomkiem}
\label{sec:uniform}
W wyniku tego krzyżowania powstaje jeden nowy osobnik, któremu przypisywane są kolejne geny rodziców, każdy z prawdopodobieństwem 50\%.

\section{Ważone z jednym potomkiem}
\label{sec:wazone}
Krzyżowanie analogiczne do powyższego, modyfikacji ulega jedynie prawdopodobieństwo otrzymania przez dziecko konkretnego genu. 
Prawdopodobieństwo to liczone jest ze wzoru:
\begin{center}
przystosowanie konkretnego rodzica / suma przystosowań obu rodziców
\end{center}

\section{Dwupunktowe z dwoma potomkami}
\label{sec:dwaDwa}
W tym typie krzyżowania chromosomy rodziców dzielone są na 3 części. Pierwsze dziecko otrzymuje początkową części materiału 
genetycznego pierwszego rodzica, środkową drugiego rodzica i końcową pierwszego rodzica. Natomiast drugie dziecko - 
początkową części materiału genetycznego drugiego rodzica, środkową pierwszego rodzica i końcową drugiego rodzica.

\section{Dwupunktowe z jednym potomkiem}
\label{sec:dwaJeden}
Krzyżowanie analogiczne do powyższego, ale w wyniku powstaje jedno dziecko. Środowa część materiału genetycznego pierwszego 
rodzica oraz początkowa i końcowa część materiału genetycznego drugiego rodzica są zaniedbywane.

\chapter{Mutacja}
\label{cha:mutation}
Zadaniem mutacji jest wprowadzenie subtelnych zmian do genotypu losowo wybranych osobników. Dzięki temu w kolejnych pokoleniach 
zachowana zostaje różnorodność w populacji, co pozwala zapobiec przedwczesnej zbieżności algorytmu. Mutacja zachodzi z 
pewnym przyjętym prawdopodobieństwem - zazwyczaj rzędu 1-5\%. Powinno być ono tak niskie, ponieważ zbyt silna mutacja przynosi 
efekt odwrotny do zamierzonego: zamiast różnicować dobre rozwiązania - niszczy je.

\section{Przy kodowaniu binarnym}
\label{sec:kodBin}
Polega na zamianie genu '1' na gen '0' (i odwrotnie).
\section{Przy kodowaniu grupowym}
\label{sec:kodGru}

\chapter{Program}
\label{cha:program}
Stworzony przez nas program "cos tam solver" został napisany w języku JAVA. Do rysowania i wizualizacji grafów użyliśmy biblioteki 
JUNG. Przy rysowaniu i tworzeniu wykresów przystosowania osobników w populacji pomocna okazała się biblioteka JFreeChart. Natomiast 
zapis wykresów do pliku umożliwiła biblioteka Batik (no chyba nie zaimplementowalismy....)!
\section{Przebieg pojedynczej iteracji}
\label{sec:singleLifeCycle}
singleLifeCycle
\section{Parametry}
\label{sec:params}
dddjjdjf
co można ustalać i co to da
\section{Wykres}
\label{sec:chart}
ffd
co jest na wykresach i dlaczego
\section{Wizualizacja}
\label{sec:visualization}
jak wygląda rozwiązanie (o niebieskich krawędziach i żółtych wierzchołkach)

\chapter{O autorach} 
\label{cha:aboutAut}
podział pracy


%% tu nie moje.
%\include{rozdzial1}
%\include{rozdzial2}



% itd.
% \appendix
% \include{dodatekA}
% \include{dodatekB}
% itd.

%\bibliographystyle{alpha}
%\bibliography{bibliografia}
%\begin{thebibliography}{1}
%
%\bibitem{Dil00}
%A.~Diller.
%\newblock {\em LaTeX wiersz po wierszu}.
%\newblock Wydawnictwo Helion, Gliwice, 2000.
%
%\bibitem{Lam92}
%L.~Lamport.
%\newblock {\em LaTeX system przygotowywania dokumentów}.
%\newblock Wydawnictwo Ariel, Krakow, 1992.
%
%\bibitem{Alvis2011}
%M.~Szpyrka.
%\newblock {\em {On Line Alvis Manual}}.
%\newblock AGH University of Science and Technology, 2011.cccccc
%\newblock \\\texttt{http://fm.ia.agh.edu.pl/alvis:manual}.
%
%\end{thebibliography}

\end{document}